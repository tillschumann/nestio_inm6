\documentclass{article}
 
\usepackage{amsmath}
\usepackage{graphicx}
\usepackage{caption}
\usepackage{subfigure}
\usepackage{epstopdf}
\usepackage[ansinew]{inputenc}
\usepackage{listings}
\usepackage{xcolor}
%\setlength{\oddsidemargin}{0cm}
%\setlength{\evensidemargin}{0cm}
%\setlength{\topmargin}{0cm}

\usepackage{hyperref}
\usepackage[normalem]{ulem}

\usepackage{a4wide}

\title{Simulating the I/O behavior of the NEST simulator using a proxy}

\author{Till Schumann, Alexander Peyser, Wolfram Schenck, Wolfgang Frings, Kay Thust, Jochen Martin Eppler}
\date{INM-6}

\begin{document}
\maketitle

Over the past 20 years, the NEST Initiative
(\url{http://www.nest-initiative.org/}) has developed the NEST
\cite{NEST} simulator for spiking neural network models. It is used in
computational neuroscience to simulate the dynamics of the interaction
between nerve cells. The systems explored with NEST range from small
networks simulated on local machines up to large brain-scale circuits
using the full capabilites of the world's leading supercomputers. To
have this flexibility, NEST is parallelized in a hybrid fashion
using threads on a compute node and MPI to communicate between the
compute nodes \cite{Plesser07}.  Storing simulation data from such a
massively parallel application efficiently during runtime is essential
for neuroscientic studies, but not a trivial task on a supercomputer
with centralized storage.

To obtain measurements from a simulation in NEST, virtual devices can
be connected to simulated nerve cells. These devices store values
associated with neurons by writing time to variables mappings such as
spiking activity or membrane voltage of nerve cells to
disk. Currently, each measurement device opens its own file using
plain C++ streams.  With an increasing number of devices and compute
nodes, writing data to disk in this way becomes the bottleneck for the
whole simulation. Due to the constant growth in simulation sizes and
therefore compute nodes, a new writing strategy has become necessary.

One solution to the problem is the use of efficient parallel I/O
libraries. It is however not clear which of the available libraries
is best. In order to assess the characteristics of different
libraries, we have implemented a \emph{proxy} which imitates the writing
behavior of NEST based on a small set of parameters such as the number
of nerve cells to record from, the number of threads and processes, or
the rate at which data is recorded.

This technique has allowed us to run benchmarks and carry out
statistical analysis of the disk writing timings and optimizations of
the interfaces without the computational costs and complexity of
running a full NEST simulation, allowing a larger number of
measurements for a larger number of parameter variations in less
time. Currently, the proxy includes experimental interfaces for plain
text, HDF5 \cite{hdf2010hierarchical} and SIONlib
\cite{frings2009scalable} which are parametrized with data from
profiling runs of real NEST simulations. The results of the benchmarks
will guide our choice towards a new I/O sub-system for NEST.

\bibliographystyle{plain}

\bibliography{./report}

\end{document}