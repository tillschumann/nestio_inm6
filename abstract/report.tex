\documentclass{article}
 
\usepackage{amsmath}
\usepackage{graphicx}
\usepackage{caption}
\usepackage{subfigure}
\usepackage{epstopdf}
\usepackage[ansinew]{inputenc}
\usepackage{listings}
\usepackage{xcolor}
%\setlength{\oddsidemargin}{0cm}
%\setlength{\evensidemargin}{0cm}
%\setlength{\topmargin}{0cm}

\usepackage{a4wide}

\title{Abstract - Development of a proxy for NEST: Simulating the disk writing behavior}
\author{Till Schumann}
\date{INM-6}

\begin{document}
   \maketitle

The Computational and Systems Neuroscience (INM-6) institute has developed over the years a simulator for spiking neural network models called NEST\cite{NEST}.
NEST simulates the biological activity and interaction of nerve cells. Therefore NEST is a hybrid parallelized application.
The complexity of the simulations reaches from small networks computed on local machines up to ones using the full workload of the world leading super computers.
Storing simulation data efficiently during runtime is a big problem on super computers.
But it is essential for analysis and studies.
\newline

For storing simulation data measurement devices can be connected virtually to nerve cells.
These measurement devices store local variables of the nerve cells to disk.
Depending on the location of the nerve cells the measurement devices are distributed unequally on the computation cluster.
At the moment each measurement device opens its own file handler via c++ standard output.
With increasing number of devices the disk writing becomes the bottleneck of the whole simulation.
In terms of future requirements a new disk writing strategy is necessary.
Therefore efficient parallel libraries should be used to reduce the writing timings.
We implemented experimental interfaces for HDF5\cite{hdf2010hierarchical} and Sionlib\cite{frings2009scalable}. 
An optimization of the library interfaces with NEST is difficult.
Even a comparison of the performance for our use cases of the libraries is difficult.
Hence we needed a tool to analyze the writing performance for different interfaces.  
\newline

We developed a proxy which imitates the writing behavior of NEST and generates benchmarks for statistical analysis of the disk writing timings.
The proxy is a toolbox for performance analysis of the interfaces.
Optimization of the library interfaces is possible without computational overhead.
Thus small modification can be tested directly.
\newline
..

\bibliographystyle{plain}

\bibliography{./report}

\end{document}